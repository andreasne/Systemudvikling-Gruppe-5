\subsection{Virksomhedsbes�g}
Combine: \\
EG neo process A/S:

\subsection{Agil process}
Process m�ssigt benytter de deres egen agile metode "Accelerate", hvilket er en tilpasning af scrum som m�ske ikke er s� vildt agil endda.

\subsubsection{Accelerate (Scrum baseret)}
Mange elementer fra agile: 
\begin{itemize}
\item produktvision
\item backlog
\item sprint
\item sprintbacklog
\item scrum m�der
\item kunder �nsker ofte mere dokumentation
\item Kvallitetssikring
\end{itemize}
      
Deres kvalitetssikring best�r i h�j grad af konstant kontakt med kunde, samt l�bende integration og test ved kunden. Derudover laves der code reviews hvor en kollega l�ser en udviklers kode igennem.                          


\subsection{Udfordringer}
En udfordring er en kundes modenhed og samarbejdsvilje eftersom EG neoprocess afh�nger meget af kommunikation og samarbejde med kunden. \\
At v�re p� forkant med Microsoft opdateringer; det sker at Microsoft laver opdateringer som kr�ver mange ressourcer at skulle s�tte sig ind i. \\


Procentdel der reelt er kodning noget i stilen af 20 % resten i er h�j krav kunde konsultation.
Forskellige l�sninger: F� 100-500 timer til 10.000 timers projekter.

\subsection{Konsulentvirksomhed  }
Sv�rt ved at leve op til agile metoder. Man er typisk p� et projekt 3-4 dage om ugen. Resten af tiden til drift p� gammel software.

\subsection{projekt varighed}
Projekter varer nogen gange 2 �r -> God relation til kunde. God efter 2 �r, det vil de helst ikke af med.


\subsection{Ansatte}
Arkitekt der er 100 \% dedikeret \\
Arkitekter: Meget bredere. \\

Forretningskonsulenter: Begr�nset viden om AX(produkt) med ved meget om virkskomheder -> kan beskrive projekter. G�r EG kan tilpasse projektet, til det, kunder vil have.(Dette ved de ikke altid selv) \\

Kunden �nsker noget nyt - men aner ikke hvad de vil have. De har ikke tid til at finde ud af det. Her bruges forretnings konsulenter. Hj�lper til at beskrive problemer, aflevere kravene til EG \\

Senere i perioden, bliver udviklere mere hjemme. \\


I arbejder agilt og i SCRUM. Scrum-master: Det har de. P� driftskunder kan de have en agil tilgang <- teams i landet. hvert team 5-10 kunder de drifter. Daglige support opgaver. Disse ryger i en "sprint backlog". Nogle teams har scrum master. Andre tager bare opgaver. \\
Et decideret projekt: S� er der SCRUM master! \\

Nogle kunder insistere p� at have ressourcer med i et team.