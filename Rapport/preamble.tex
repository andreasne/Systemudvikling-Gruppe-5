\usepackage[danish]{babel} % Forklarer at dokumentet er dansk
\usepackage[latin1]{inputenc} % Forklarer hvilket charset er dokumentet skrevet i
\usepackage{graphicx} % Manage external pictures
\usepackage{pdfpages} % This package simplifies the insertion of external multi-page PDF or PS documents.
\usepackage{listings} % to insert programming code within the document. Many languages are supported and the output can be customized.
\usepackage{float} % Bruges blandt andet til H i figures
\usepackage{fancyhdr} % change header and footer of any page of the document.
\usepackage[pdfborder={0 0 0 0}]{hyperref} % Indstter url-adresser og ref-links korrekt. Disse kan vises som url adressen eller bare tekst og er klik-bare i pdf dokumentet. \href{URL}{text}
\usepackage{verbatim} % improves the verbatim environment
\usepackage{rotating} % Bruges til at vende teksten vertikalt
\usepackage{framed} % Benyttes til at lave bokse om diverse ting.
\usepackage{multirow} % Benyttes til avancerede tabeller
\usepackage{amssymb} % it adds new symbols in to be used in math mode.
\usepackage{color} % it adds support for colored text
\usepackage{natbib} % gives additional citation options and styles
\usepackage{subfigure} % addfunctionality for subfigures

\usepackage{tikz}
\usepackage[none]{hyphenat}

% All of this crap is used for the parsing table DO NOT DELETE!
\def\inputGnumericTable{} 
\usepackage{array}
\usepackage{longtable}
\usepackage{calc}
\usepackage{multirow}
\usepackage{hhline}
\usepackage{ifthen}
% End of parsing table crap

%added inorder to make pipe literals withinb tables in the ebnf
\usepackage[T1]{fontenc}

\lstset{
	language=java,
	keywordstyle=\bfseries\ttfamily\color[rgb]{0,0,1},
	identifierstyle=\ttfamily,
	commentstyle=\color[rgb]{0.133,0.545,0.133},
	stringstyle=\ttfamily\color[rgb]{0.627,0.126,0.941},
	showstringspaces=false,
	basicstyle=\small,
	numberstyle=\footnotesize,
	numbers=left,
	stepnumber=1,
	numbersep=10pt,
	tabsize=2,
	breaklines=true,
	prebreak = \raisebox{0ex}[0ex][0ex]{\ensuremath{\hookleftarrow}},
	breakatwhitespace=false,
	aboveskip={1.5\baselineskip},
	frameround=fftt,
	frame=shadowbox,
	columns=fixed,
	upquote=true,
	extendedchars=true,
	breaklines=true
%	alsoletter={'},
%	morekeywords={'}
% frame=single,
% backgroundcolor=\color{lbcolor},
}
\newcommand\XOR{\mathbin{\char`\^}}
\usepackage{textcomp}
