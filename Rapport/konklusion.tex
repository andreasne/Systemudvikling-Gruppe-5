Efter at have lavet b�de ide udvikling, kvalitetssikring, risici analyse og planl�gning udviklede vi gennem 3 sprints en mindre del af appen Food4Thought og lavede unit testing til database adgangen. Til udviklingen benyttede vi en agil tilgang, samt en blanding af scrum og xp praktikker alt efter behovet. Vores risici analyse forinden for p�begyndelsen af udviklingen kunne godt have v�ret mere grundig, men vi mener ikke at dette havde gjort den store forskel for udviklingsprocessen. \\
Vi forventer at det ville tage 3 m�neder m�neders tid, opdelt i sprints p� 2-3 uger, at f�rdigg�re vores app til en endelig release. \\

Vores sprints var meget korte og det blev stresset i forhold til at have noget klar til hvert review efter hvert sprint, hvilket ikke blev bedre af at vi kun var en gruppe p� 3 personer. \\
Det var en udfordring at opn� konsistens mellem produkt backlog og sprint backlog, samt at definere vores user stories optimalt.
Vi burde have igangsat et prototype projekt forinden vi gik i gang med det egentlige udviklingsprojekt og lavet l�ngere spikes til at blive mere kendt med teknologien. I sprintsene kom det til at f�les en smule akavet at arbejde med en bestemt process blot for at g�re det, specielt i forbindelse med Android, som vi manglede erfaring med.