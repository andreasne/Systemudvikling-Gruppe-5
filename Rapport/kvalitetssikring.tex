\subsection*{Kvalitetsscenarier}
Her er de forskellige kvalitetsscenarier vi forventer kan have relevans for systemet. \\

\includegraphics[scale=0.60]{includes/billeder/kvalitetsscenarietabel.png}

\textsc{Termer:} \\
Safety/sikkerhed: \\
Brugeren: Kan brugeren komme til skade \\

Security/sikkerhed: \\
Systemet: Kan systemet �del�gges \\
Brugeren: Er brugerens data sikret. \\

Reliability/p�lidelighed: \\
Systemet: Er systemet stabilt. \\
Appen: Kan brugeren stole p� informationen. \\

Resilience/modstandsdygtigt: \\
Systemet: Kan komme op at k�re hurtigt igen efter problemer. \\

Robustness/robust: \\
Systemet: Stabilt og crasher ikke. \\
Database: Kan h�ndtere foresp�rgsler. \\

Testability/testbart: \\
App: Funktionaliteten er testbar som sikrer usability. \\

Adaptability/tilpasseligt: \\
Systemet: Kan tilpasses i forhold til udvikling af nyt system eller kravs�ndringer. \\

Modularity/modul�rt:\\ 
Systemet: N�dvendighed for adaptability og scalability.\\

Complexity/komplekst: \\
Systemet: FoodMap kompleksitet, algoritme. Modvirker adaptability? \\

Portability/overf�rlighed:\\ 
App: Kan portes til andre platforme. \\


Usability/brugbarhed: \\
App: Nemt at bruge. Vigtigt \\

Reusability/genbrugelighed: \\
App: videre udvikling, diabetiker, fitness app. \\
Database: genbrug til anden platform. \\

%Vigtighed af kvalitetsscenarier: \\
%- Resilience: **** \\
%- Testability: ***** \\
%- Adaptability: **** \\
%- Modularity: **** \\
%- Compleksitet: *** \\
%- Portability: ** \\
%- Usability: ***** \\
%- Reusability: *

\newpage

\subsection*{Kravscenerier}
Vi har lavet kravscenerier til de 2 vigtigste stories: "Bruger vil gerne have ingrediens forslag" og "Bruger vil gerne have liste med opskrifter". \\  

\textbf{Bruger vil gerne have ingrediens forslag} \\
\includegraphics[scale=0.30]{includes/billeder/kravsscenerie_foodwheel.png} \\
I dette scenarie vil brugeren gerne have forslag til ingredienser. Han interagerer s� med genstanden som er en android app i milj�et Android 4.3 og f�r som respons vist grafiske bobler der indeholder forslag til ingredienser indenfor maks 5 sekunder. Den korte responstid er vigtig for at brugeren f�r en god oplevelser og ikke bliver ut�lmodig.

\textbf{Bruger vil gerne have liste med opskrifter} \\
\includegraphics[scale=0.30]{includes/billeder/kravsscenerie_opskriftliste.png} \\
I dette scenarie vil brugeren gerne have en liste med forslag til opskrifter. Han interagerer s� med genstanden som er en android app i milj�et Android 4.3 og f�r som respons vist en liste af opskrifter indenfor maks 5 sekunder. Den korte responstid er vigtig for at brugeren f�r en god oplevelser og ikke bliver ut�lmodig.