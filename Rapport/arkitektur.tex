Vi har valgt at arkitekturen b�r v�re et sammenspil mellem klienten og en eller flere server(e). N�dvendigheden af en lokal database forekommer da appen skal kunne k�re uden internet adgang, idet at afh�ngighed af internet adgang ville kunne �del�gge brugeroplevelsen. Samtidigt er det n�dvendigt med en eller flere server(e) med separat database, for at den lokale database kan udvides og opdateres, uden at brugeren skal til at hente en ny opdateret version af appen.

\includegraphics[scale=0.40]{includes/billeder/arkitektur.png} \\
Et par problemstillinger der opst�r for vores arkitektur er; hvor ofte vi b�r synkronisere lokale databaser med server databaserne. Hvor store m� opdateringer v�re f�r de begynder at genere brugeren samt b�r opdateringer foretages automatisk eller b�r brugeren selv manuelt st� for det. \\ \\
En l�sningen kunne v�re p� forh�nd at advare brugeren i terms of use for appen at opdateringer som standard foretages automatisk, og brugeren skal v�re klar p� at opdateringer kan have en vis st�rrelse og derfor kan koste en del data, samt at databasen opdateres hvert kvartal. Brugeren kan s� f� muligheden for selv manuelt at lave �ndringer p� indstillingerne gennem en simpel menu.