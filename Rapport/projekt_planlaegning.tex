\subsection*{Risici}
Vi har valgt at analysere vores project i forhold bohms model, samt at lave risici analyse p� forskellige muligheder mht. teknologi heriblandt: Android, Windows phone og mobile web. For hver enkelt teknologi skal overvejes specielt i forhold vores erfaring med den samt relevans ift. m�let. \\

\subsection*{Bohms model}
% efterf�lgende skal opdateres
\includegraphics[scale=0.32]{includes/billeder/bohmsmodel.png}

% efterf�lgende skal fjernes n�r vores egen analyse nedenuden er opdateret til tilpas lignende format.
\includegraphics[scale=0.32]{includes/billeder/risici_analyse.png}

\begin{center}
\line(1,0){450}
\end{center}

\subsection*{Risici analyse}
\subsubsection*{Android phone}
Omgivelser (Environment), E-risici mht. projektets omgivelser.
\begin{itemize}
\item E-Tech. Usikkerhed mht. teknologien: 2 
\begin{itemize}
\item Android ide: 2
\item Versionstyring: 3
\item Emulator: 1
\end{itemize}

\item E-Coord. Interessenter der skal koordineres: 0
\item E-Cmplx. Systemets kompleksitet: 2
\begin{itemize}
\item FoodMap funktionalitet: 2
\item API: 1
\end{itemize}

\item A-Scale. St�rrelse og kritiskhed: 0
\item A-YAGNI. Simpelt design eller YAGNI: 1
\item A-Churn. Udskiftning af udviklere: 0
\item A-Skill. Ikke nok uddannelse af folk til at arbejde agilt: 3
\item P-Change. Mange �ndringer: 1
\item P-Speed. Behov for hurtige resultater: 2
\item P-Emerge. Uklare krav: 1
\item P-Skill. Mangel p� specialister: 3
\end{itemize}

\begin{center}
\line(1,0){450}
\end{center}

\begin{center}
\line(1,0){450}
\end{center}

\subsubsection*{Windows phone} 
\begin{itemize}
\item E-Tech. Usikkerhed mht teknologien: 1
\item E-Coord. Interessenter der skal koordineres: 1
\item E-Cmplx. Systemets kompleksitet: 0
\item A-Scale. St�rrelse og kritiskhed: 0
\item A-YAGNI. Simpelt design elller YAGNI: 4
\item A-Churn. Udskiftning af udviklere: 0
\item A-Skill. Ikke nok uddannelse af folk til at arbejde agilt: 3
\item P-Change. Mange �ndringer: 0
\item P-Speed. Behov for hurtige resultater: 1
\item P-Emerge. Uklare krav: 1
\item P-Skill. Mangel p� specialister: 3
\end{itemize}

\begin{center}
\line(1,0){450}
\end{center}

Database

\newpage

\subsection*{Projektplanl�gning og styring}
Iterationsplan og releaseplan, burndown chart

\subsection*{Kvalitetssikring og test}
Vi benytter pair programming og dokumentation. \\
Vi har valgt ikke at benytte Unit testing, alfa/betatests og continuous integration fordi ...

\subsection*{Deployment og konfigurationsstyring}
Omkring at ops�tte og teste softwaren kundens systemer. Kan g�res med forskelligt mellemrum afh�ngigt af metode.

\newpage

\subsection*{Velocity}

Vi forventer at have 4 timer om dagen per mand i gruppen. Dvs. 8 mandetimer om dagen med 2 hold. \\
4 dage f�rste uge. \\
4 dage anden uge. \\
5 dage tredje uge. \\

5 storypoint (standard) tager 8 mandetimer \\
5/8 = 0,625 storypoint per mandetime \\

%dette skal �ndres...
40 mande timer i en iteration
40*0,625 = 25 storypoints per uge
			5 storypoints per dag \\

4 sprints i alt.

\newpage

\subsection*{Sprint 0}
Sprint 0 benyttede vi til at lave produktbacklog, iterationsplan og burndown. Samt at lave spike p� interfacet til foodmappet. \\
Vi skulle ogs� have lavet spike p� database men n�ede det ikke. \\ 
I �vrigt var Bjarke var ikke tilstede hverken onsdag, torsdag og fredag pga. Gr�n IT.

\subsection*{Sprint 1 med XP}
Til sprint 1 havde vi lavet et burndown som viste storypoints per mandedage. \\
Vi benyttede pair-programming og fokuserede p� de 2 user-stories: \\
Hvor der indgik database design, oprettelse af tabeller og gui samt generel funktionalitet til foodmap ... \\
Vi l�b ind i et problem med at vores userstory for databasen var for bredt defineret, hvilket resulterede i at vi ikke kunne tegne vores burndown ned. Vi valgte derfor i starten af sprint 2 at dele den op. Derudover delte vi ogs� andre store userstories op...

\newpage

Backloggen s� i sprint 1 s�dan ud:
%link til at genere tabeller: http://www.tablesgenerator.com/
\begin{table}[h]
\begin{tabular}{llll}
Userstory                                                                                                                                                                    & \begin{tabular}[c]{@{}c@{}}Story-\\ points\end{tabular} & \begin{tabular}[c]{@{}c@{}}Mande-\\ timer\end{tabular} & Prioritet                \\ \hline
\multicolumn{1}{|l}{\begin{tabular}[c]{@{}c@{}}Kunden vil gerne have flere forslag til\\ ingredienser frem p� food map, ved at\\ trykke p� den midterste knap.\end{tabular}} & \multicolumn{1}{|l}{3}                                  & \multicolumn{1}{|l}{3}                                 & \multicolumn{1}{|l|}{1}  \\ \hline
\multicolumn{1}{|l}{\begin{tabular}[c]{@{}c@{}}Kunden vil have et animeret foodmap, \\ hvor ingredienser popper ud fra den\\ eksisterende ingrediensbobbel.\end{tabular}}    & \multicolumn{1}{|l}{}                                   & \multicolumn{1}{|l}{}                                  & \multicolumn{1}{|l|}{2}  \\ \hline
\multicolumn{1}{|l}{\begin{tabular}[c]{@{}c@{}}Kunden vil gerne kunne fjerne og tilf�je\\ ingredienser i food map.\end{tabular}}                                             & \multicolumn{1}{|l}{}                                   & \multicolumn{1}{|l}{}                                  & \multicolumn{1}{|l|}{3}  \\ \hline
\multicolumn{1}{|l}{\begin{tabular}[c]{@{}c@{}}Kunden vil gerne have en liste over forskellige\\ opskrifter som er udvalgt af \\ de anf�rte ingredienser.\end{tabular}}      & \multicolumn{1}{|l}{}                                   & \multicolumn{1}{|l}{}                                  & \multicolumn{1}{|l|}{4}  \\ \hline
\multicolumn{1}{|l}{\begin{tabular}[c]{@{}c@{}}Kunden vil gerne have hurtig s�geforslag \\ til ingredienser. (Performance)\end{tabular}}                                     & \multicolumn{1}{|l}{}                                   & \multicolumn{1}{|l}{}                                  & \multicolumn{1}{|l|}{5}  \\ \hline
\multicolumn{1}{|l}{\begin{tabular}[c]{@{}c@{}}Kunden vil gerne have listet opskrifter med\\ billeder, beskrivelse, ingredienser og \\ tilberedning.\end{tabular}}           & \multicolumn{1}{|l}{}                                   & \multicolumn{1}{|l}{}                                  & \multicolumn{1}{|l|}{6}  \\ \hline
\multicolumn{1}{|l}{\begin{tabular}[c]{@{}c@{}}Kunden vil gerne kunne f� et hurtigt overblik\\  over indk�bs- og pr�perationsliste ved opskrift.\end{tabular}}               & \multicolumn{1}{|l}{}                                   & \multicolumn{1}{|l}{}                                  & \multicolumn{1}{|l|}{7}  \\ \hline
\multicolumn{1}{|l}{\begin{tabular}[c]{@{}c@{}}Kunden vil gerne have en menu til navigering\\ i appen.\end{tabular}}                                                         & \multicolumn{1}{|l}{}                                   & \multicolumn{1}{|l}{}                                  & \multicolumn{1}{|l|}{8}  \\ \hline
\multicolumn{1}{|l}{\begin{tabular}[c]{@{}c@{}}Kunden vil gerne kunne gemme indk�bsliste\\ i indk�bsoversigt.\end{tabular}}                                                  & \multicolumn{1}{|l}{}                                   & \multicolumn{1}{|l}{}                                  & \multicolumn{1}{|l|}{9}  \\ \hline
\multicolumn{1}{|l}{\begin{tabular}[c]{@{}c@{}}Kunden vil gerne kunne gemme opskrifter i\\ favoritter.\end{tabular}}                                                         & \multicolumn{1}{|l}{}                                   & \multicolumn{1}{|l}{}                                  & \multicolumn{1}{|l|}{10} \\ \hline
\multicolumn{1}{|l}{\begin{tabular}[c]{@{}c@{}}Kunden vil gerne have en s�rskilt indk�bsliste,\\  som kan tilg�s fra hovedmenuen.\end{tabular}}                              & \multicolumn{1}{|l}{}                                   & \multicolumn{1}{|l}{}                                  & \multicolumn{1}{|l|}{11} \\ \hline
\end{tabular}
\end{table}


\subsection*{Sprint 2}
Vi lavede en ny burndown som viste mandetimer per mandedage. \\
Vi splittede vores userstories og produktbacklog bedre op, samt lavede nye estimater i b�de storypoint og mandetimer p� hver enkelt userstory. Derudover satte vi ogs� tidsestimater p� vores tasks for bedre at v�re i stand til at afslutte tasks og at kunne tegne vores burndown ned. \\
Sprintet kom i h�j grad til at handle om userstories for at lave database samt kobling af database og foodmap.

\subsection*{Product backlog}
Backlog lavet den 9.12.2013 i starten af sprint 2. \\
\includegraphics[scale=0.60]{includes/billeder/productbacklog_sprint2.png}

\subsection*{Sprint 3}