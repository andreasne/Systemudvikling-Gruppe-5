\section*{Planl�gning og styring}
Til planl�gning og styring benytter vi iterationsplan med userstories og tasks samt en releaseplan og et burndown chart. \\
Vi har brugt scrum master for alle sprints, hvilket Bjarke har st�et for. Dette gjorde vi selv i sprint 1 som egentligt var baseret p� xp, men valgte at g�re det fordi det gav mening. Vi har valgt ikke at have en product owner idet at det ikke virkede effektivt i s� lille en gruppe og det giver ikke mening at productowner er en del af udviklingsteamet, da han dermed bliver mere tilb�jelig til at godkende �ndringer. Accepttest har vi skrevet bag p� vores userstories, hvor vi s� for at godkende dem har haft en af gruppemedlemmerne til at tjekke at det er er opfyldt, samt at koden for den bestemte funktionalitet har kunnet kompileres og eksekveres uden at g� ned.

\subsection*{Teknologi beslutninger}
Versionsstyring: GIT \\
Teknologi: Android og SQLite \\ 
Rapport skrivning: LaTex \\
IDE: Eclipse

\subsection*{Kvalitetssikring og test}
Vi har benyttet par programmering i det omfang det ellers har givet mening. \\
I det f�rste 3 sprints har vi valgt ikke at benytte unit testing da vi, p� grund af teknologien, er pressede tidsm�ssigt til at have funktionalitet f�rdigt til produkt reviews. \\ 
Automatisk l�bende integration og test ville blive for tidskr�vende at s�tte op, men pusher ofte via GIT efter at have sikret at det kan kompileres og eksekveres.

\newpage

\subsection*{Velocity}
Vi forventer at have 4 timer om dagen per mand i gruppen. Dvs. 8 mandetimer om dagen med 2 hold. \\
4 dage f�rste uge. \\
4 dage anden uge. \\
3 dage tredje uge. \\

5 storypoint (standard) tager 8 mandetimer \\
5/8 = 0,625 storypoint per mandetime \\

%dette skal �ndres...
40 mande timer i en iteration
40*0,625 = 25 storypoints per uge
			5 storypoints per dag \\

4 sprints i alt.

\subsection*{Revurderet Velocity}
F�lgende har vi fundet ud af vores velocity er lavere pga. manglende erfaring og teknisk niveau for nogle i gruppen.
Dette resulterer i at vi ikke helt har resourcerne for to hold men n�rmere 1,5.\\

Vi forventer at have 4 timer om dagen per mand i gruppen. Dvs. 6 mandetimer om dagen med 1,5 hold.\\

24 mande timer i en iteration\\
24*0,625 = 15 storypoints per uge\\
			3,75 storypoints per dag \\

\newpage

\section*{Sprint 0}
Sprint 0 benyttede vi til at lave produkt backlog, iterationsplan og burndown. Samt at lave spike p� interfacet til foodmappet. \\
Vi skulle ogs� have lavet spike p� database men n�ede det ikke. \\ 
I �vrigt var Bjarke var ikke tilstede hverken onsdag, torsdag og fredag pga. Gr�n IT.

\newpage

\section*{Sprint 1 med XP}
\subsection*{Velocity}
I sprint 1 forventede vi at have 4 timer om dagen per mand i gruppen. Dvs. 8 mandetimer om dagen med 2 hold med
4 arbejdsdage. \\
Vi regnede os til at 5 storypoint tager 8 mandetimer \\
Det svarer til: 5 storypoints / 8 mandetimer = 0,625 storypoint per mandetime \\

32 mande timer i en iteration. \\
32 mandetimer * 0,625 = 20 storypoints per uge, dvs. 5 storypoints per dag. Dog tog vi et ekstra storypoint med for ugen.

\subsection*{Planl�gning}
Til sprintet afbillede vi vores burndown som en funktion af storypoints per mandedage. \\

% inds�t burndown her

\subsubsection*{Product backlog} 
Product backloggen s� i sprint 1 s�dan ud: \\
%link til at genere tabeller: http://www.tablesgenerator.com/
\includegraphics[scale=0.60]{includes/billeder/productbacklog_sprint1.png}

\subsubsection*{Sprint backlog}
I sprintet fokuserede vi p� 2 userstories:
\begin{itemize}
\item Brugeren vil gerne have flere forslag til ingredienser frem p� food map, ved at trykke p� den midterste knap.
\item Brugeren vil gerne have adgang til database.
\end{itemize}

Disse userstories omhandlede tasks med: 
\begin{itemize}
\item database design
\item oprettelse af tabeller
\item gui samt generel funktionalitet til foodmap
\end{itemize}

\subsection*{Xp praktikker}
I sprint 1 benyttede vi XP praktikkerne: stand-up meeting, planning poker, par programmering, kollektivt kode ejerskab, refactoring, kodestandarder, story board, metafor og simpelt design. \\

Vi lavede planning poker med storypoints til at vurdere st�rrelsen af userstories for senere hen at udregne vores velocity. \\
Vi arbejdede med par programmering i 2 grupper, hvor vi ogs� i h�j grad havde den af grupperne med en mand i til at unders�ge teknologien. Vi fors�gte s� vidt muligt at f� alle ind over koden hvor vi i forvejen havde aftalt at bruge Java camelcasing samt at refactorer koden med KISS som m�l.
Derudover vi benyttede dom�ne model og database schema til design for at f� enighed om samt overblik over dom�net. 

Mht. kvalitetssikring valgte vi ikke at benytte test-first eller unit testing eftersom vi p� dette tidspunkt ikke rigtigt havde noget relevant at teste p� samt at vi prioriterede funktionalitet.



\subsection*{Produkt review}
Til reviewet i starten af sprint 2 pr�senterede vi vores grafiske brugergr�nseflade vi havde p� nuv�rende tidspunkt ikke 
% inds�t screenshot her.
 
\subsubsection*{Konklusion af review}
At vi havde et lovende interface med statisk funktionalitet, men uden database adgang.


\subsection*{Retrospektive}
Sprint 1 har ikke forl�bet som forventet vi har ikke f�et et br�ndt et eneste storypoint af, da vi ikke har haft for store userstories og vi har ikke lagt op til at kunne br�nde p� tasks. Endvidere har vi ikke n�et at programmerer det vi havde forventet. Dette har v�ret pga. vores manglende erfaring indenfor android og vi har derfor brugt for lang tid p� at fors�ge at s�tte database op. Selvom vi har beregnet spikes ind i vores userstories har vi ikke lavet et realistisk estimat af dette.\\

Vi kan derfor konkluderer at sprint 1 har fejlet pga. for store userstories, der er sv�re at repr�senterer p� vores burndown og pga. d�rlig estimering af userstories, som er et m�n fra vores d�rlige erfaring med android. Vi har estimeret ops�tning af database ud fra vores tidligere erfaring med ops�tning af database i nye platformer.


\subsubsection*{Hvad gik godt}
I sprint 1 benyttede vi XP praktikkerne: stand-up meeting, planning poker, par programmering, kollektivt kode ejerskab, refactoring, kodestandarder, l�bende integration, story board, kanban begr�nsninger, metafor og simpelt design. \\


\textsf{Planning Poker:}\\
Vi har brugt planning poker til at estimerer de enkelte userstories. I vores situation kom vi til at undervurdere vores userstories men det mener vi ikke selve praktikken skal have skyld for.\\ \\
\textsf{Par programmering:}\\
Vi har i h�j grad brugt par programmering hvor vi har siddet i 2 grupper. 1 med 2 personer og 1 med 1 person.\\ \\
\textsf{Kollektivt kode ejerskab:}\\
Vi har brugt kollektivt kodeejerskab dog ikke helt som xp foreskriver, hvor man f.eks. bytter p� et bestemt tidspunkt p� dagen, men i den forstand at vi er en lille gruppe og vi har alle m�ttet s�tte os ind i tingene for at kunde kode i Android.\\ \\
\textsf{Stand-up meeting:}\\
Vi har holdt standup meeting hver morgen for at vurdere hvor langt vis hver is�r er n�et i kodningsprocessen. Derefter har vi bedre kunnet vurdere hvorvidt det har v�ret muligt at blive f�rdig med de enkelte tasks.\\ \\
\textsf{Refactoring:}\\
Refactoring har vi brugt i stor grad idet de enkelte sprints ogs� har v�ret en l�ringsprocess. Bla. har vi m�ttet �ndre vores tilgang til hvordan databasen skulle s�ttes op. I begyndelsen lavede vi databasen i selve koden, men vi blev senere klar over at vi blev n�dt til at lave insert i databasen i selve koden og at dette ville skabe k�mpe uoverskuelige store filer. Derfor blev vi n�dt til at s�tte databasen op i SQLite f�rst, ligge den ind i vores android projekt og kalde vores data derfra.\\ \\
\textsf{Metafor:}\\
Vi har undervejs blevet enige om hvilken terminologi der skulle bruge i vores projekt. Bla. har vi defineret noget af dette p� vores dom�nemodel og database diagram.\\ \\
\textsf{Story board:}\\
Vi har brugt story board for at have overblik over hvor langt vi er n�et med de enkelte tasks. \\ \\
\textsf{Simpelt Design:}\\
Vi har kun implementeret den n�dvendige kode.\\ \\
\textsf{Kodestandarder:}\\
Vi har defineret kodestandarder i sprint 0 for at kunne fremvise p�n kode.\\


\subsubsection*{Hvad gik mindre godt}
\textsf{Opsplitning af user stories:} \\
Vores user stories skal splittes mere og bedre op, s� vi er bedre istand til at tegne burndown ned. \\

\textsf{Burndown kunne laves bedre:} \\
Vores Burndown var lavet med storypoints som funktion af dage. \\ \\

\textsf{Korrigere for frav�r}\\
Vi fik ikke ordentligt taget hensyn til frav�r og bare lidt frav�r g�r en meget stor forskel n�r vi er s� f� og har s� f� resurser. \\ \\

\textsf{Andet:} \\
Vi lavede ikke test-first - Idet at det ikke var muligt da vi ikke er erfarne med Android. \\ \\

\subsubsection*{Ting vi forts�tter med} 
Vi forts�tter med at benytte pair-progamming til de mere komplekse tasks idet at det er et godt v�rkt�j til at s�tte os ind i ny teknologi sammen.

\subsubsection*{Hvad �ndrer vi til n�ste sprint}
Burndown sammensat af timer og dage, i stedet for point. \\
Opdeling af user stories. \\
Vil lave burndown af tasks istedet for user stories. \\
Vores accept tests.

\subsection*{Sprint Konklusion}

\newpage

\subsection*{Sprint 2}
Ud fra sprint 1 har vi erfaret at for store userstories giver en uklar pr�sentation af fremskridt p� vores burndown. Vi startede derfor sprint 2 med at dele vores userstories op i mindre stories. Derefter lavede vi planning poker igen for at f� et mere n�jagtigt estimat af hvor lang tid de forskellige userstories vil tage. Endvidere for at f� et mere n�jagtig burndown begyndte vi at br�nde ned i mandetimer istedet for userstories og s�tte mandetimer p� de enkelte tasks.
Selve sprintet kommer i h�j grad til at handle om userstories for at lave database samt koblingen mellem databasen og foodmap.

\subsection*{Velocity}
Vi har ikke �ndret p� vores velocity fra sprint 1, da vi mener at vi ikke fik br�ndt ned p� burndown pga. for store userstories og d�rlig estimering.

I sprint 2 forventede vi derfor at have 4 timer om dagen per mand i gruppen. Dvs. 8 mandetimer om dagen med 2 hold med
4 arbejdsdage. \\
Vi regnede os til at 5 storypoint (standard) tager 8 mandetimer \\
Det svarer til: 5 storypoints / 8 mandetimer = 0,625 storypoint per mandetime \\

32 mande timer i en iteration. \\
32 mandetimer * 0,625 = 20 storypoints per uge, dvs. 5 storypoints per dag. Dog tog vi et ekstra storypoint med for ugen.

\subsection*{Planl�gning}
 
\subsubsection*{Product backlog}
Backlog lavet den 9.12.2013 i starten af sprint 2. \\
\includegraphics[scale=0.60]{includes/billeder/productbacklog_sprint2.png}

\subsubsection*{Sprint backlog}
I sprint 2 fokuserede vi p� f�lgende userstories: \\
Database kobling, dummy data til databasen, algoritme. \\
Samt database tabeller og basic gui som vi satte til 0 mandetimer eftersom delene var f�rdige efter at have opdelt dem, da de var for store. 

\subsection*{Xp og scrum praktikker}
I sprint 2 benyttede vi XP praktikkerne: stand-up meeting, planning poker, par programmering, kollektivt kode ejerskab, refactoring, kodestandarder, metafor og simpelt design. \\

Vi lavede planning poker med storypoints til at vurdere st�rrelsen af userstories for senere hen at udregne vores velocity. \\
Vi arbejdede med par programmering i 2 grupper, hvor vi ogs� i h�j grad havde den af grupperne med en mand i til at unders�ge teknologien. Vi fors�gte s� vidt muligt at f� alle ind over koden hvor vi i forvejen havde aftalt at bruge Java camelcasing samt at refactorer koden med KISS som m�l.
Derudover vi benyttede dom�ne model og database schema til design for at f� enighed om samt overblik over dom�net.

Mht. kvalitetssikring valgte vi ikke at benytte test-first eller unit testing eftersom vi p� dette tidspunkt ikke rigtigt havde noget relevant at teste p� samt at vi prioriterede funktionalitet. Dermed er vores eneste form for kvalitetssikring par programmering.

\subsection*{Forl�b}

\subsection*{Produkt review}
Til produkt reviewet var det planen at vi ville vise vores gui med database adgang, men emuleratoren fejlede og pr�sentationen faldt til bunden. Derfor endte det med at vi kun kunne vise selve databasen med data. 

\subsubsection*{Konklusion af review}
Vi skulle have v�ret lidt bedre forberedt til reviewet, dermed det m�ske v�ret muligt at undg� problemer med emulatoren og kunden havde f�et et bedre indtryk.

\subsection*{Retrospektive}
I sprintet blev vi f�rdige med databaseadgangen. \\
Vi havde overvurderet hvor mange resurser vi havde til r�dighed.

\subsubsection*{Ting der gik godt}

\subsubsection*{Mindre godt}

\subsubsection*{Hvad �ndrer vi til n�ste sprint}

\newpage

\newpage

\subsection*{Sprint 3}
\subsection*{Velocity}
Lav velocity blah blah...

\subsection*{Planl�gning}

\subsubsection*{Product backlog}
Til product backloggen har vi i dette sprint tilf�jet en userstory med unit testing, samt at vi har �ndret prioriteten for diverse stories. \\
\includegraphics[scale=0.60]{includes/billeder/productbacklog_sprint3.png}

\subsubsection*{Sprint backlog}
Unit testing af database adgang og algoritme.



\subsection*{Xp og scrum praktikker}
Par programmering, simpelt design,

\subsection*{Produkt review}
Viste vores funktionalitet, samt vores unit test.

\subsubsection*{Konklusion af review}
N�ede kun unit testing.

\subsection*{Retrospektive}
Godt: \\
Vi fik lavet unit test og dermed fik vi lidt mere kvalitetssikring ind over vores projekt. \\

Kunne have v�ret bedre: \\
Alt for kort sprint med de 3 mandedage og derfor for meget pres p� for at have noget klar til reviewet. Is�r iforhold til at have noget visuelt klar at kunne vise. \\
Vi havde lidt meget parprogrammering pga. vores fokus p� test. 

Hvad �ndrer vi til n�ste sprint: \\
Vi var sprint 3 det sidste sprint, men ellers havde vi prim�rt fors�ge at lave et l�ngere sprint hvor vi arbejdede lidt mere spredt.


\section*{Konklusion p� sprints}
Vores sprints var for korte og der var for meget stress p� i forhold til at have noget klar til hvert review. \\
hvad har v�ret godt og d�rligt i sprintsene...
