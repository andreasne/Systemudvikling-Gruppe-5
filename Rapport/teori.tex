

\subsection*{Rapport indhold:}
Max halvdelen til teori og virksomhedsbes�g \\
40 sider max \\
Reflektion over process brugt i projektet \\
Teoretiske del:

\subsection*{Metoder, deres klassifikation og anvendelse:}
modul 1,2,3 og 4 \\ \\

Agile: xp, scrum og kanban \\
Plandrevne: UP og vandfald

\subsection*{Extreme Programming}
Extreme programming er en agil udviklingsmetoder der kom frem i starten af 90'erne og var opfundet af en mand  ved navn Kent Beck. Metoden best�r af et netv�rk af v�rdier og praktikker som bygger p� sund fornuft.
Mange firmaer l�gger deres programeringsprocess om til agile udviklingsmetoder. Tit bliver det en blanding mellem de mere plandrevne og agile processer da ikke alle kunder f�ler sig sikrer ved at bruge en ren agil metode som xp eller scrum.




\subsection*{Valg af metode og risikoanalyse:}
Modul 5 \\

Kritiske projekt faktorer (Boehm) \\
St�rrelse \\
Kritisk st�rrelse af fejlen \\
Dynamik i omgivelser (krav) \\
Udviklere \\
Kultur