EG NeoProcess A/S, tidligere (thy:data), og ligger i Aalborg �st p� Alfred Nobelsvej 21E i Novi forskerparken. Virksomheden er i h�j grad en konsulentvirksomhed som tilbyder specialtilpassede modul l�sninger indenfor Dynamics AX (tidligere Axapta). Procent m�ssig er det noget i stilen af 20 \% som koder, mens resten i h�j grad besk�ftiger sig med kunde konsultation. Virksomheden tilbyder blandt andet brug af deres egen agile udviklingsprocess "Accelerate", som er en tilpasning af Scrum.


\subsection*{Typer af ansatte}
Virksomheden har mange forskellige typer ansatte. \\

Arkitekt der er 100 \% dedikeret \\
Arkitekter: Meget bredere. \\

Forretningskonsulenter: \\ 
Begr�nset viden om AX (produkt), med ved meget om virksomheder s� kan beskrive udviklingsprojekter for kunden. G�r EG kan tilpasse projektet til hvad kunder vil have. \\
Specielt n�r kunden �nsker noget nyt - men ikke ved hvad de vil have og ikke har ressourcer til at finde ud af det bruges forretnings konsulenter som hj�lper til at beskrive problemer, aflevere kravene til EG \\

Typisk bliver udviklere ogs� sendt ud til at kunden, is�r tidligt i processen. Senere i udviklingsprocessen, n�r kravene i h�jere grad er p� plads, bliver udviklere s� mere hjemme. \\



\subsubsection*{Accelerate}
Accelerate indeholde mange elementer fra de agile metoder:
\begin{itemize}
\item produkt vision
\item sprints
\item produkt backlog
\item sprint backlog
\item scrum m�der
\item scrum master
\item kvalitetssikring
\end{itemize}

Projekter har meget forskellig varighed afh�ngigt af hvilken l�sning kunden �nsker, men der er f� 100-500 timer til 10.000 timers projekter. \\
 
P� driftskunder kan de have en agil tilgang og teams i landet. Hvert team har typisk 5-10 kunder som de st�r for drift support til. Der er daglige support opgaver, som logges i en "sprint backlog". Nogle teams har en scrum master mens andre bare tager opgaver, men for deciderede projekter benyttes scrum master. \\
 
Kvalitetssikring best�r i h�j grad af konstant kontakt med kunde, samt l�bende integration og test ved kunden. Derudover laves der code reviews hvor en kollega l�ser en udviklers kode igennem.                           


\subsection*{Udfordringer}
En udfordring er kundes modenhed og samarbejdsvilje: EG neoprocess afh�nger meget af god kommunikation og samarbejde med kunden. \\
Det at v�re p� forkant med Microsoft opdateringer; det sker at Microsoft laver opdateringer som kr�ver mange ressourcer at skulle s�tte sig ind i. \\
Nogle kunder insistere p� at have ressourcer med i et team, hvilket NeoProcess ikke altid er lige glade for.

\subsubsection*{Konsulentvirksomhed}
Sv�rt ved at leve op til agile metoder. Man er typisk p� et projekt 3-4 dage om ugen. Resten af tiden til drift p� gammel software.
Der er mange kunder som �nsker mere dokumentation end NeoProcess's udviklingsmetode Accelerate producerer.

\subsubsection*{Projekt varighed}
Projekter varer nogen gange 2 �r -> God relation til kunde. God efter 2 �r, det vil de helst ikke af med.